\begin{abstract}
As Artificial Intelligence (AI) systems reshape industries, organizations must decide how best to deploy finite human cognition while enhancing AI-assisted workflows. This paper develops a unified framework for allocating limited cognitive resources between direct human tasks and the refinement of AI systems. The model treats both forms of effort as part of a single resource budget, highlighting the fundamental trade-off between human execution and AI-driven enhancements of tasks under standard utility assumptions. By positing a strictly concave production function for AI contributions, the analysis proves that an interior solution emerges in which direct labor and AI improvements are jointly optimized. At this interior solution, the marginal gains from investing in AI match those from relying on human expertise, ensuring that organizations capture the advantages of automation without compromising essential oversight. In contrast, if the AI production function is linear or convex, corner solutions exist that either dedicate nearly all resources to AI or exclude it altogether, potentially reducing all human oversight as AI capabilities continue to advance. Second-order conditions confirm the uniqueness and stability of this interior equilibrium, highlighting the practical contexts in which AI and human expertise achieve complementary gains. These findings offer actionable insights for managers, engineers, and researchers seeking to integrate human potential with increasingly sophisticated AI.
\end{abstract}
\noindent \textbf{Key Words:} Artificial Intelligence, Resource Allocation, Constrained Optimization, Diminishing Returns, Strict Concavity, Bounded Rationality


\noindent \textbf{AMS subject classifications:} 90C25, 91B38, 68T20, 46N10
\newpage
